\documentclass[Conclusion]{subfiles}
\begin{document}
\section{Conclusion}
\label{sec:Conclusion}
This paper is about how Naxsi influences the performance of the \mbox{Nginx} web server. The performance of Naxsi is looked at from two perspectives. First, Naxsi is isolated by making Nginx return only \verb+HTTP 200 OK+ responses, and thus allowing Nginx to handle a high amount of requests per second. Second, the performance is measured with a more realistic scenario. This is done by hosting a Wordpress on a back-end server. In both cases, a baseline performance measurement is conducted to find the maximum performance when Naxsi is not compiled into the Nginx web server. When Nginx returns only \verb+HTTP 200 OK+ responses, the Nginx server is capable of handling up to $11,973$ requests per second when $380$ concurrent connections are used. The back-end server that is hosting the Wordpress website is capable of handling $190$ concurrent connections, with a response time of $\approx 30 ms$ per request. For measuring the impact of the performance of Naxsi, the number of URL parameters is incremented from 1 to 20. First, all parameters have valid content and Naxsi allows the request to be further processed. With the \verb+HTTP 200 OK+ responses, the number of requests per second drops with $\approx 35\%$ when 20 URL parameters are used, compared to when no URL parameters are used. The response time for serving a Wordpress web site increases with $0.8 ms$ when 20 URL parameters are used, compared to when no parameters are used. When Naxsi encounters an invalid URL parameter, there is a small performance loss for denying the request. In our setup, Naxsi is configured to return an \verb+HTTP 403+ error code when it encounters an invalid request. Compared to when 10 valid URL parameters are used, Naxsi is capable of handling $\approx 10,000$ requests per second. But, when the 10th URL parameters has invalid content, then Naxsi drops to $\approx 5,000$ requests per second. Invalid URL parameters have a different result when hosting the Wordpress website, because processing a Wordpress web page takes considerable longer than simply returning an \verb+HTTP 403+ error code.

The usage of Naxsi greatly depends on the application that needs to be protected by Naxsi. When a Wordpress website is hosted, then there is a very little performance loss compared to hosting the website without the protection of Naxsi. In this case, Naxsi shows very little impact on the system resources and the response time only slightly increases. When several thousands of requests per second need to be processed by Naxsi, then a clearer performance loss becomes visible. 
\end{document}

