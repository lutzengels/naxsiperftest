\documentclass[Methods]{subfiles}
\begin{document}
\section{Methods}
\label{sec:Methods}

In order to find the overhead Naxsi has when it is actively used as a \ac{WAF}, it is important to create a baseline reference to compare the results to. First, the performance is measured when there is processing of any application data taken place. This is done by returning HTTP 200 OK message to the client. This way there is a minimal overhead and this makes it possible to isolate Naxsi as much as possible. Second, the performance is measure with a default Wordpress installation. In both cases, the performance is measured first with Naxsi being disabled and later with Naxsi enabled.

Before both scenarios are discussed in more depth, it is important to understand how the basic configuration is setup. A configuration for standard web hosting that is seen quite often, is to separate the web hosting services on a functional basis. First, there is the web server, this the front-end from a client's perspective. Second, there is application layer. The application layer takes care of most of the processing power that is needed to process all the data. Third, there is the data layer, which is often a database server. It is not necessary to separate these layers over different servers. Depending on the requirements, it is possible to host all services on one server. In order to do a performance measurement on one of these services, it is important that not all services run on the same server. Therefore, in this setup, every layer has a dedicated server. In the next section the setup is discussed more in depth.

\subsection{Experimental setup}

Figure \ref{fig:Experimental setup} shows the setup that is used during the experiments. Table \ref{tab:Experimental infrastructure} gives a short description of each server and the service(s) that run(s) on it. Server01 is the front-end server, but it will also act as a software router for basic communication with the servers behind it. Server02 processes all the application data, which for a large part consists of the processing of PHP code. The hardware specification and the software that is used, can be found in appendix \ref{sec:Experimental setup}.

\begin{figure}[h]
\caption{Experimental setup}
\centering
\includegraphics[scale=0.4] {images/infrastructure.png}
\label{fig:Experimental setup}
\end{figure}

\begin{table}[h]
\caption{Experimental infrstructure}
\begin{tabular}{|p{2,5cm}|p{3,5cm}|p{5cm}|}
\hline
\textbf{Hostname} & \textbf{Service} & \textbf{Short description} \\ \hline
server01 & Nginx + Naxsi & Front-end server and router \\ \hline
server02 & Nginx + Fastcgi & Application layer \\ \hline
server03 & MySQL & Data layer \\ \hline
server04 & Benchmark tools & Performance measurement \\ \hline
server05 & Collectd & Performance data collector \\ \hline
\end{tabular}
\label{tab:Experimental infrastructure}
\end{table}

\subsection{Performance measurement tools}
A variety of performance measurement tools (also called benchmarking tools) exist. However not all of them are suited to perform tests in this specific experimental setup while also not influencing the measurement. This section discusses a selection of tools. The sections of those that have in fact been used, specify the way they were set up.

\subsubsection{Apache Benchmark}
\emph{Apache Benchmark} is a web server benchmarking tool developed by Apache initially to test Apache web server installations. However, it is not limited to Apache web servers, as it  It simulates HTTP requests, where 
During the testing phase \emph{Apache Benchmark} turned out not to be flexible enough to generate requests that yield measurements


\subsubsection{Httperf}
%\emph{Httperf} is another web server performance measurement tool by Hewlet-Packard. As described in section~\ref{sec:Baseline performance measurement} it was used to determine the maximum number of concurrent connections that naxsi is able to handle. Details like the used parameters can be found in Appendix~\ref{sec:Measurement results}.


\subsubsection{Autobench}
\emph{Autobench} is a Perl script that instruments \emph{httperf} to benchmark web servers. It offers extensive options, amongst which the ability to stepwise increase concurrent connections or the number of request. This results in graphs that visualize bottlenecks/tipping points. In the course of this project it has been used to determine the threshold of the maximum number of concurrent connections.~\ref{sec:Baseline performance measurement}


\subsubsection{Collectd}
Collectd is a daemon for unix-based operating systems that gathers performance statistics. It has a modular design, meaning that the different kind of statistics (like e.g. cpu-load or network-usage) are en- or disabled by toggling the respective plugins and thereby minimizing resource usage. Furthermore it only handles the data collection and storage hereof, leaving out logic to create graphs, whilst storing data in \ac{RRD} files. Programs like RRDtool in turn can easily create graphs from the RRD-files. It is written in the fast and low impact programming language C~\cite{prechelt2000empirical} and designed to be run on e.g. embedded devices.
As to influence the experimental setup as little as possible Collectd appeared to be the right choice.

Collectd is configured on all servers of the experimental setup. \emph{server01} to \emph{server04} are configured as clients, collecting their own local performance data. \emph{server05} also collects its own local performance data, but also receives those from the clients. Making use of the \emph{collection3} web-based front-end RRDtool is utilized to graph the collected data.
\\
For detailed configuration please refer to Appendix B.2.


\subsubsection{Raw Naxsi performance test}

\subsubsection{Naxsi performance test with back-end server}
Based on the \ac{CMS} popularity measurements of March 1st by w3techs \footnote{\url{http://w3techs.com/technologies/overview/content_management/all}}, Wordpress is chosen as the web application. According to their survey, 17,4 \% of all web servers run Wordpress and it holds 54,6 \% of the CMS market share. Wordpress has, by far, the greatests market share and is therefore a popular target for cyber criminals.
\end{document}

