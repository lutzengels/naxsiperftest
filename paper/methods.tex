\documentclass[Methods]{subfiles}
\begin{document}
\section{Methods}
\label{sec:Methods}

\subsection{Performance measurement tools}
\begin{itemize}
\item httperf
\item siege
\item Apache Benchmark
\item Collectd
\end{itemize}

\subsection{Measurement metrics}
\begin{itemize}
\item Response time
\item CPU load (all servers)
\item Memory usage (all servers)
\item IO (all servers)
\item Network utilization (all servers)
\end{itemize}

\subsection{Experiment infrastructure}
As there is no standard set of rules for performance testing web servers, the described definition aims at being, what in medicine is called, a gold standard test~\cite{wacholder1993validation}. This comprises that the test environment as close as possible mimics a real life situation. \\ 
In order to The following things are vital to performance testing:

\begin{figure}[h]
\caption{Experimental setup}
\centering
\includegraphics[scale=0.4] {images/infrastructure.png}
\label{fig:Experimental setup}
\end{figure}

\subsection{Expermimental setups}

\subsubsection{Raw Naxsi performance test}

\subsubsection{Naxsi performance test with back-end server}
Based on the \ac{CMS} popularity measurements of March 1st by w3techs \footnote{\url{http://w3techs.com/technologies/overview/content_management/all}}, Wordpress is chosen as the web application. According to their survey, 17,4 \% of all web servers run Wordpress and it holds 54,6 \% of the CMS market share. Wordpress has, by far, the greatests market share and is therefore a popular target for cyber criminals.




\end{document}

