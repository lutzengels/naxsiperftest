\documentclass[Configuration]{subfiles}
\begin{document}
\section{Experimental setup}
\label{sec:Experimental setup}
% TODO: figure infrastructure

\subsection{Hardware}

\subsubsection{server01}
\begin{table}[h]
\centering
\begin{tabular}{p{2cm} p{9cm} }
\textbf{Brand} & Dell \\
\textbf{Model} & PowerEdge R210 \\
\textbf{CPU} & Intel(R) Xeon(R) CPU L3426 @ 1.87GHz (2x)\\
\textbf{Memmory} & 8 GB RAM, 1066 MHz (4x 2GB) \\
\textbf{Disk} & 500 GB, SATA 3.0 Gbps (2x)\\
\textbf{Interfaces} &  Integrated 10/100/1000 Mbps NIC (2x) \\
 & Broadcom NetXtreme II BCM5716 1000Base-T (C0) PCI Express Dual-port \\ 
\textbf{OS} & Debian Squeeze 6.0.7 \\
\end{tabular}
\end{table}

\subsubsection{server02}
\begin{table}[h]
\centering
\begin{tabular}{p{2cm} p{9cm} }
\textbf{Brand} & SunMicro \\
\textbf{Model} & X7DBT/X7DGT \\
\textbf{CPU} & Intel(R) Xeon(R) CPU E5420 @ 2.50GHz (2x)\\
\textbf{Memmory} & 16 GB RAM, 1333 MHz (8x 2GB) \\
\textbf{Disk} & 160 GB, SATA 3.0 Gbps (2x)\\
\textbf{Interfaces} &   Intel(R) PRO/1000 Network (2x) \\
\textbf{OS} & Debian Squeeze 6.0.7 \\
\end{tabular}
\end{table}

\subsubsection{server03}
\begin{table}[h]
\centering
\begin{tabular}{p{2cm} p{9cm} }
\textbf{Brand} & SunMicro \\
\textbf{Model} & X7DBT/X7DGT \\
\textbf{CPU} & Intel(R) Xeon(R) CPU E5420 @ 2.50GHz (2x)\\
\textbf{Memmory} & 16 GB RAM, 1333 MHz (8x 2GB) \\
\textbf{Disk} & 160 GB, SATA 3.0 Gbps (2x)\\
\textbf{Interfaces} &   Intel(R) PRO/1000 Network (2x) \\
\textbf{OS} & Debian Squeeze 6.0.7 \\
\end{tabular}
\end{table}

\subsubsection{server04}
\begin{table}[h]
\centering
\begin{tabular}{p{2cm} p{9cm} }
\textbf{Brand} & SunMicro \\
\textbf{Model} & X7DBT/X7DGT \\
\textbf{CPU} & Intel(R) Xeon(R) CPU E5420 @ 2.50GHz (2x)\\
\textbf{Memmory} & 16 GB RAM, 1333 MHz (8x 2GB) \\
\textbf{Disk} & 160 GB, SATA 3.0 Gbps (2x)\\
\textbf{Interfaces} &   Intel(R) PRO/1000 Network (2x) \\
\textbf{OS} & Debian Squeeze 6.0.7 \\
\end{tabular}
\end{table}

\subsection{Configuration}

\subsubsection{Time synchronization}
All servers update their local time automatically with the use of cronjob

\begin{lstlisting}[frame=single,caption=cronjob -l ,backgroundcolor=\color{gray},breaklines=true,numbers=left,]
@hourly /usr/sbin/ntpdate 0.nl.pool.ntp.org 1.nl.pool.ntp.org 2.nl.pool.ntp.org 3.nl.pool.ntp.org
\end{lstlisting}

\subsubsection{sever01}

\begin{lstlisting}[frame=single,caption=/etc/udev/rules.d/70-persistent-net.rules,backgroundcolor=\color{gray},breaklines=true,numbers=left,]
KERNEL=="eth*", ATTR{address}=="b8:ac:6f:8b:81:bd", NAME="eth0"
KERNEL=="eth*", ATTR{address}=="b8:ac:6f:8b:81:be", NAME="eth1"
KERNEL=="eth*", ATTR{address}=="00:15:17:74:7b:3c", NAME="eth2"
KERNEL=="eth*", ATTR{address}=="00:15:17:74:7b:3d", NAME="eth3"
\end{lstlisting}

\begin{lstlisting}[frame=single,caption=/etc/network/interfaces,backgroundcolor=\color{gray},breaklines=true,numbers=left,]
auto lo
iface lo inet loopback

auto lo
iface lo inet loopback

# The primary network interface
allow-hotplug eth0
iface eth0 inet dhcp

auto eth1
iface eth1 inet static
        address 10.1.2.1
        netmask 255.255.255.0
        network 10.1.2.0
        broadcast 10.1.2.255

auto eth2
iface eth2 inet static
        address 10.1.3.1
        netmask 255.255.255.0
        network 10.1.3.0
        broadcast 10.1.3.255

auto eth3
iface eth3 inet static
        address 10.1.4.1
        netmask 255.255.255.0
        network 10.1.4.0
        broadcast 10.1.4.255
\end{lstlisting}

\begin{lstlisting}[frame=single,caption=/etc/resolv.conf,backgroundcolor=\color{gray},breaklines=true,numbers=left,]
# sysctl -p
net.ipv4.ip_forward = 1
\end{lstlisting}

\begin{lstlisting}[frame=single,caption=/etc/resolv.conf,backgroundcolor=\color{gray},breaklines=true,numbers=left,]
nameserver 145.100.96.11
nameserver 145.100.96.22
\end{lstlisting}

\begin{lstlisting}[frame=single,caption=/etc/network/interfaces,backgroundcolor=\color{gray},breaklines=true,numbers=left,]
#!/bin/bash

# Allow internet for internal network
iptables -t nat -A POSTROUTING -o eth0 -s 10.1.0.0/16 ! -d 10.0.0.0/8 -j SNAT --to-source 145.100.104.61

# Allow ssh to reach internal network
iptables -t nat -A PREROUTING -i eth0 -p tcp --dport 1002 -j DNAT --to-destination 10.1.1.2:22
iptables -t nat -A PREROUTING -i eth0 -p tcp --dport 1003 -j DNAT --to-destination 10.1.2.2:22
iptables -t nat -A PREROUTING -i eth0 -p tcp --dport 1004 -j DNAT --to-destination 10.1.3.2:22
\end{lstlisting}

\begin{lstlisting}[frame=single,caption=Compiling nginx+naxsi,backgroundcolor=\color{gray},breaklines=true,numbers=left,]
wget http://nginx.org/download/nginx-1.2.7.tar.gz
wget http://naxsi.googlecode.com/files/naxsi-core-0.48.tgz
tar xvzf naxsi-core-0.48.tgz
tar xvzf nginx-1.2.7.tar.gz
cd nginx-1.2.7/
./configure --conf-path=/etc/nginx/nginx.conf \
--add-module=../naxsi-core-0.48/naxsi_src/ \
--error-log-path=/var/log/nginx/error.log \
--http-client-body-temp-path=/var/lib/nginx/body \
--http-fastcgi-temp-path=/var/lib/nginx/fastcgi \
--http-log-path=/var/log/nginx/access.log \
--http-proxy-temp-path=/var/lib/nginx/proxy \
--lock-path=/var/lock/nginx.lock \
--pid-path=/var/run/nginx.pid \
--with-http_ssl_module \
--without-mail_pop3_module \
--without-mail_smtp_module  \
--without-mail_imap_module  \
--without-http_uwsgi_module \
--without-http_scgi_module \
--with-ipv6 \
--prefix=/usr
make
make install
mkdir /etc/nginx/sites-enabled
\end{lstlisting}

\begin{lstlisting}[frame=single,caption=/etc/nginx/nginx.conf,backgroundcolor=\color{gray},breaklines=true,numbers=left,]
TO BE INCLUDED WHEN WORKING
\end{lstlisting}

\begin{lstlisting}[frame=single,caption=/etc/nginx/sites-enabled/default,backgroundcolor=\color{gray},breaklines=true,numbers=left,]
TO BE INCLUDED WHEN WORKING
\end{lstlisting}

\begin{lstlisting}[frame=single,caption=/etc/nginx/nbs.rules,backgroundcolor=\color{gray},breaklines=true,numbers=left,]
TO BE INCLUDED WHEN WORKING
\end{lstlisting}


\subsubsection{server02}

\begin{lstlisting}[frame=single,caption=/etc/network/interfaces,backgroundcolor=\color{gray},breaklines=true,numbers=left,]
auto lo
iface lo inet loopback

auto eth0
iface eth0 inet static
        address 10.1.2.2
        netmask 255.255.255.0
        network 10.1.2.0
        broadcast 10.1.2.255
        gateway 10.1.2.1

auto eth1
iface eth1 inet static
        address 10.2.2.2
        netmask 255.255.255.0
        network 10.2.2.0
        broadcast 10.2.2.255
\end{lstlisting}

\begin{lstlisting}[frame=single,caption=/etc/resolv.conf,backgroundcolor=\color{gray},breaklines=true,numbers=left,]
nameserver 145.100.96.11
nameserver 145.100.96.22
\end{lstlisting}

\subsubsection{server03}

\begin{lstlisting}[frame=single,caption=/etc/network/interfaces,backgroundcolor=\color{gray},breaklines=true,numbers=left,]
auto lo
iface lo inet loopback

auto eth0
iface eth0 inet static
        address 10.1.3.2
        netmask 255.255.255.0
        network 10.1.3.0
        broadcast 10.1.3.255
        gateway 10.1.3.1

auto eth1
iface eth1 inet static
        address 10.2.2.3
        netmask 255.255.255.0
        network 10.2.2.0
        broadcast 10.2.2.255
\end{lstlisting}

\begin{lstlisting}[frame=single,caption=/etc/resolv.conf,backgroundcolor=\color{gray},breaklines=true,numbers=left,]
nameserver 145.100.96.11
nameserver 145.100.96.22
\end{lstlisting}

\subsubsection{server04}

\begin{lstlisting}[frame=single,caption=/etc/network/interfaces,backgroundcolor=\color{gray},breaklines=true,numbers=left,]
auto lo
iface lo inet loopback

auto eth0
iface eth0 inet static
        address 10.1.4.2
        netmask 255.255.255.0
        network 10.1.4.0
        broadcast 10.1.4.255
        gateway 10.1.4.1
\end{lstlisting}

\begin{lstlisting}[frame=single,caption=/etc/resolv.conf,backgroundcolor=\color{gray},breaklines=true,numbers=left,]
nameserver 145.100.96.11
nameserver 145.100.96.22
\end{lstlisting}
\end{document}

