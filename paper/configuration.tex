\documentclass[Configuration]{subfiles}
\begin{document}
\section{Experimental setup}
\label{sec:Experimental setup}
% TODO: figure infrastructure

\subsection{Hardware}

\subsubsection{server01}
\begin{table}[h]
\centering
\begin{tabular}{p{2cm} p{9cm} }
\textbf{Brand} & Dell \\
\textbf{Model} & PowerEdge R210 \\
\textbf{CPU} & Intel(R) Xeon(R) CPU L3426 @ 1.87GHz (2x)\\
\textbf{Memmory} & 8 GB RAM, 1066 MHz (4x 2GB) \\
\textbf{Disk} & 500 GB, SATA 3.0 Gbps (2x)\\
\textbf{Interfaces} &  Integrated 10/100/1000 Mbps NIC (2x) \\
 & Broadcom NetXtreme II BCM5716 1000Base-T (C0) PCI Express Dual-port \\ 
\textbf{OS} & Debian Squeeze 6.0.7 \\
\end{tabular}
\end{table}

\subsubsection{server02}
\begin{table}[h]
\centering
\begin{tabular}{p{2cm} p{9cm} }
\textbf{Brand} & SunMicro \\
\textbf{Model} & X7DBT/X7DGT \\
\textbf{CPU} & Intel(R) Xeon(R) CPU E5420 @ 2.50GHz (2x)\\
\textbf{Memmory} & 16 GB RAM, 1333 MHz (8x 2GB) \\
\textbf{Disk} & 160 GB, SATA 3.0 Gbps (2x)\\
\textbf{Interfaces} &   Intel(R) PRO/1000 Network (2x) \\
\textbf{OS} & Debian Squeeze 6.0.7 \\
\end{tabular}
\end{table}

\subsubsection{server03}
\begin{table}[h]
\centering
\begin{tabular}{p{2cm} p{9cm} }
\textbf{Brand} & SunMicro \\
\textbf{Model} & X7DBT/X7DGT \\
\textbf{CPU} & Intel(R) Xeon(R) CPU E5420 @ 2.50GHz (2x)\\
\textbf{Memmory} & 16 GB RAM, 1333 MHz (8x 2GB) \\
\textbf{Disk} & 160 GB, SATA 3.0 Gbps (2x)\\
\textbf{Interfaces} &   Intel(R) PRO/1000 Network (2x) \\
\textbf{OS} & Debian Squeeze 6.0.7 \\
\end{tabular}
\end{table}

\subsubsection{server04}
\begin{table}[h]
\centering
\begin{tabular}{p{2cm} p{9cm} }
\textbf{Brand} & SunMicro \\
\textbf{Model} & X7DBT/X7DGT \\
\textbf{CPU} & Intel(R) Xeon(R) CPU E5420 @ 2.50GHz (2x)\\
\textbf{Memmory} & 16 GB RAM, 1333 MHz (8x 2GB) \\
\textbf{Disk} & 160 GB, SATA 3.0 Gbps (2x)\\
\textbf{Interfaces} &   Intel(R) PRO/1000 Network (2x) \\
\textbf{OS} & Debian Squeeze 6.0.7 \\
\end{tabular}
\end{table}

\subsection{Configuration}

\subsubsection{Time synchronization}
All servers update their local time automatically with the use of cronjob

\begin{lstlisting}[frame=single,caption=cronjob -l ,backgroundcolor=\color{gray},breaklines=true,numbers=left,]
@hourly /usr/sbin/ntpdate 0.nl.pool.ntp.org 1.nl.pool.ntp.org 2.nl.pool.ntp.org 3.nl.pool.ntp.org
\end{lstlisting}

\subsubsection{sever01}

\begin{lstlisting}[frame=single,caption=/etc/udev/rules.d/70-persistent-net.rules,backgroundcolor=\color{gray},breaklines=true,numbers=left,]
KERNEL=="eth*", ATTR{address}=="b8:ac:6f:8b:81:bd", NAME="eth0"
KERNEL=="eth*", ATTR{address}=="b8:ac:6f:8b:81:be", NAME="eth1"
KERNEL=="eth*", ATTR{address}=="00:15:17:74:7b:3c", NAME="eth2"
KERNEL=="eth*", ATTR{address}=="00:15:17:74:7b:3d", NAME="eth3"
\end{lstlisting}

\begin{lstlisting}[frame=single,caption=/etc/network/interfaces,backgroundcolor=\color{gray},breaklines=true,numbers=left,]
auto lo
iface lo inet loopback

auto lo
iface lo inet loopback

# The primary network interface
allow-hotplug eth0
iface eth0 inet dhcp

auto eth1
iface eth1 inet static
        address 10.1.2.1
        netmask 255.255.255.0
        network 10.1.2.0
        broadcast 10.1.2.255

auto eth2
iface eth2 inet static
        address 10.1.3.1
        netmask 255.255.255.0
        network 10.1.3.0
        broadcast 10.1.3.255

auto eth3
iface eth3 inet static
        address 10.1.4.1
        netmask 255.255.255.0
        network 10.1.4.0
        broadcast 10.1.4.255
\end{lstlisting}

\begin{lstlisting}[frame=single,caption=/etc/resolv.conf,backgroundcolor=\color{gray},breaklines=true,numbers=left,]
# sysctl -p
net.ipv4.ip_forward = 1
\end{lstlisting}

\begin{lstlisting}[frame=single,caption=/etc/resolv.conf,backgroundcolor=\color{gray},breaklines=true,numbers=left,]
nameserver 145.100.96.11
nameserver 145.100.96.22
\end{lstlisting}

\begin{lstlisting}[frame=single,caption=/etc/network/interfaces,backgroundcolor=\color{gray},breaklines=true,numbers=left,]
#!/bin/bash

iptables -F INPUT
iptables -t nat -F

# Allow internet for internal network
iptables -t nat -A POSTROUTING -o eth0 -s 10.1.0.0/16 ! -d 10.1.0.0/8 -j SNAT --to-source 145.100.104.61

# Allow ssh to reach internal network
iptables -t nat -A PREROUTING -i eth0 -p tcp --dport 1002 -j DNAT --to-destination 10.1.2.2:22
iptables -t nat -A PREROUTING -i eth0 -p tcp --dport 1003 -j DNAT --to-destination 10.1.3.2:22
iptables -t nat -A PREROUTING -i eth0 -p tcp --dport 1004 -j DNAT --to-destination 10.1.4.2:22

# Block http from the internet
iptables -A INPUT -i eth0 -p tcp --dport 80 -j DROP
\end{lstlisting}

\begin{lstlisting}[frame=single,caption=Compiling nginx+naxsi,backgroundcolor=\color{gray},breaklines=true,numbers=left,]
wget http://nginx.org/download/nginx-1.2.7.tar.gz
wget http://naxsi.googlecode.com/files/naxsi-core-0.48.tgz
tar xvzf naxsi-core-0.48.tgz
tar xvzf nginx-1.2.7.tar.gz
cd nginx-1.2.7/
./configure --conf-path=/etc/nginx/nginx.conf \
--add-module=../naxsi-core-0.48/naxsi_src/ \
--error-log-path=/var/log/nginx/error.log \
--http-client-body-temp-path=/var/lib/nginx/body \
--http-fastcgi-temp-path=/var/lib/nginx/fastcgi \
--http-log-path=/var/log/nginx/access.log \
--http-proxy-temp-path=/var/lib/nginx/proxy \
--lock-path=/var/lock/nginx.lock \
--pid-path=/var/run/nginx.pid \
--with-http_ssl_module \
--without-mail_pop3_module \
--without-mail_smtp_module  \
--without-mail_imap_module  \
--without-http_uwsgi_module \
--without-http_scgi_module \
--with-ipv6 \
--prefix=/usr
make
make install
mkdir /etc/nginx/sites-enabled
mkdir -p /var/lib/nginx/body
\end{lstlisting}

\begin{lstlisting}[frame=single,caption=/etc/nginx/nginx.conf,backgroundcolor=\color{gray},breaklines=true,numbers=left,]
orker_processes  8;

events {
    worker_connections  1024;
}

http {
    include       mime.types;
    default_type  application/octet-stream;

    sendfile        on;

    keepalive_timeout  65;

    include /etc/nginx/sites-enabled/*;
}
\end{lstlisting}

\begin{lstlisting}[frame=single,caption=/etc/nginx/sites-enabled/with\_naxsi.test.nl,backgroundcolor=\color{gray},breaklines=true,numbers=left,]
upstream with_naxsi.test.nl {
  server 10.1.2.2;
}

server {
  listen                80;

  server_name           with_naxsi.test.nl;

  location / {
    proxy_pass          http://with_naxsi.test.nl;
    proxy_redirect      off;
    proxy_set_header    Host             $host;
    proxy_set_header    X-Real-IP        $remote_addr;
    proxy_set_header    X-Forwarded-For  $proxy_add_x_forwarded_for;
  }
}
\end{lstlisting}

\begin{lstlisting}[frame=single,caption=/etc/nginx/nbs.rules,backgroundcolor=\color{gray},breaklines=true,numbers=left,]
TO BE INCLUDED WHEN WORKING
\end{lstlisting}

\subsubsection{server02}

%Mainly based on \url{http://library.linode.com/web-servers/nginx/php-fastcgi/debian-6-squeeze}

\begin{lstlisting}[frame=single,caption=/etc/network/interfaces,backgroundcolor=\color{gray},breaklines=true,numbers=left,]
auto lo
iface lo inet loopback

auto eth0
iface eth0 inet static
        address 10.1.2.2
        netmask 255.255.255.0
        network 10.1.2.0
        broadcast 10.1.2.255
        gateway 10.1.2.1

auto eth1
iface eth1 inet static
        address 10.2.2.2
        netmask 255.255.255.0
        network 10.2.2.0
        broadcast 10.2.2.255
\end{lstlisting}

\begin{lstlisting}[frame=single,caption=/etc/resolv.conf,backgroundcolor=\color{gray},breaklines=true,numbers=left,]
nameserver 145.100.96.11
nameserver 145.100.96.22
\end{lstlisting}

\begin{lstlisting}[frame=single,caption=necessary packages,backgroundcolor=\color{gray},breaklines=true,numbers=left,]
# apt-get install nginx spawn-fcgi php5-common php5-mysql php5-xmlrpc php5-cgi php5-curl php5-gd php5-cli  php-apc php-pear php5-dev php5-imap php5-mcrypt
\end{lstlisting}

\begin{lstlisting}[frame=single,caption=/etc/php5/cgi/php.ini,backgroundcolor=\color{gray},breaklines=true,numbers=left,]
# echo "extension=mysql.so" >> /etc/php5/cgi/php.ini
# echo "extension=mysqli.so" >> /etc/php5/cgi/php.ini
\end{lstlisting}

\begin{lstlisting}[frame=single,caption=Wordpress 3.5.1 installation,backgroundcolor=\color{gray},breaklines=true,numbers=left,]
# mkdir -p /srv/www/with_naxsi.test.nl/logs
# cd /srv/wwww/with_naxsi.test.nl/
# wget http://wordpress.org/latest.tar.gz
# tar zxvf latest.tar.gz 
# chown -R www-data:www-data /srv/www/with_naxsi.test.nl
\end{lstlisting}

\begin{lstlisting}[frame=single,caption=/etc/nginx/sites-enabled/with\_naxsi.test.nl,backgroundcolor=\color{gray},breaklines=true,numbers=left,]
server {
    server_name with_naxsi.test.nl;
    root /srv/www/with_naxsi.test.nl/wordpress;

    location / {
        index  index.php;
    }

    location ~ \.php$ {
        include /etc/nginx/fastcgi_params;
        fastcgi_pass unix:/var/run/php-fastcgi/php-fastcgi.socket;
        fastcgi_index index.php;
        fastcgi_param SCRIPT_FILENAME /srv/www/with_naxsi.test.nl/wordpress$fastcgi_script_name;
    }
}
\end{lstlisting}

\begin{lstlisting}[frame=single,caption=/usr/bin/php-fastcgi (needs chmod +x /usr/bin/php-fastcgi),backgroundcolor=\color{gray},breaklines=true,numbers=left,]
#!/bin/bash

FASTCGI_USER=www-data
FASTCGI_GROUP=www-data
SOCKET=/var/run/php-fastcgi/php-fastcgi.socket
PIDFILE=/var/run/php-fastcgi/php-fastcgi.pid
CHILDREN=6
PHP5=/usr/bin/php5-cgi

/usr/bin/spawn-fcgi -s $SOCKET -P $PIDFILE -C $CHILDREN -u $FASTCGI_USER -g $FASTCGI_GROUP -f $PHP5
\end{lstlisting}

\begin{lstlisting}[frame=single,caption=/etc/nginx/nginx.conf,backgroundcolor=\color{gray},breaklines=true,numbers=left,]
user www-data;
worker_processes  8;

events {
    worker_connections  1024;
}

http {
    include       mime.types;
    default_type  application/octet-stream;

    sendfile        on;

    keepalive_timeout  65;

    access_log off;

    include /etc/nginx/sites-enabled/*;
}
\end{lstlisting}

\begin{lstlisting}[frame=single,caption=/etc/init.d/php-fastcgi,backgroundcolor=\color{gray},breaklines=true,numbers=left,]
#!/bin/bash

PHP_SCRIPT=/usr/bin/php-fastcgi
FASTCGI_USER=www-data
FASTCGI_GROUP=www-data
PID_DIR=/var/run/php-fastcgi
PID_FILE=/var/run/php-fastcgi/php-fastcgi.pid
RET_VAL=0

case "$1" in
    start)
      if [[ ! -d $PID_DIR ]]
      then
        mkdir $PID_DIR
        chown $FASTCGI_USER:$FASTCGI_GROUP $PID_DIR
        chmod 0770 $PID_DIR
      fi
      if [[ -r $PID_FILE ]]
      then
        echo "php-fastcgi already running with PID `cat $PID_FILE`"
        RET_VAL=1
      else
        $PHP_SCRIPT
        RET_VAL=$?
      fi
  ;;
    stop)
      if [[ -r $PID_FILE ]]
      then
        kill `cat $PID_FILE`
        rm $PID_FILE
        RET_VAL=$?
      else
        echo "Could not find PID file $PID_FILE"
        RET_VAL=1
      fi
  ;;
    restart)
      if [[ -r $PID_FILE ]]
      then
        kill `cat $PID_FILE`
        rm $PID_FILE
        RET_VAL=$?
      else
        echo "Could not find PID file $PID_FILE"
      fi
      $PHP_SCRIPT
      RET_VAL=$?
  ;;
    status)
      if [[ -r $PID_FILE ]]
      then
        echo "php-fastcgi running with PID `cat $PID_FILE`"
        RET_VAL=$?
      else
        echo "Could not find PID file $PID_FILE, php-fastcgi does not appear to be running"
      fi
  ;;
    *)
      echo "Usage: php-fastcgi {start|stop|restart|status}"
      RET_VAL=1
  ;;
esac
exit $RET_VAL
\end{lstlisting}

\begin{lstlisting}[frame=single,caption=starting services,backgroundcolor=\color{gray},breaklines=true,numbers=left,]
chmod +x /etc/init.d/php-fastcgi
update-rc.d php-fastcgi defaults
/etc/init.d/php-fastcgi start
/etc/init.d/nginx start
\end{lstlisting}

\subsubsection{server03}

\begin{lstlisting}[frame=single,caption=/etc/network/interfaces,backgroundcolor=\color{gray},breaklines=true,numbers=left,]
auto lo
iface lo inet loopback

auto eth0
iface eth0 inet static
        address 10.1.3.2
        netmask 255.255.255.0
        network 10.1.3.0
        broadcast 10.1.3.255
        gateway 10.1.3.1

auto eth1
iface eth1 inet static
        address 10.2.2.3
        netmask 255.255.255.0
        network 10.2.2.0
        broadcast 10.2.2.255
\end{lstlisting}

\begin{lstlisting}[frame=single,caption=/etc/resolv.conf,backgroundcolor=\color{gray},breaklines=true,numbers=left,]
nameserver 145.100.96.11
nameserver 145.100.96.22
\end{lstlisting}

\begin{lstlisting}[frame=single,caption=MySQL Server 5.1,backgroundcolor=\color{gray},breaklines=true,numbers=left,]
# apt-get install mysql-server
\end{lstlisting}

\begin{lstlisting}[frame=single,caption=MySQL configuration,backgroundcolor=\color{gray},breaklines=true,numbers=left,]
# sed -i 's/127.0.0.1/10.2.2.3/' /etc/mysql/my.cnf
# /etc/init.d/mysql restart

# mysql -u root -p
Enter password: 
mysql> CREATE DATABASE wordpress;
Query OK, 1 row affected (0.01 sec)

mysql> GRANT ALL PRIVILEGES ON wordpress.* TO 'naxsi'@'10.2.2.2' IDENTIFIED BY 'naxsi';
Query OK, 0 rows affected (0.00 sec)
\end{lstlisting}

\subsubsection{server04}

\begin{lstlisting}[frame=single,caption=/etc/network/interfaces,backgroundcolor=\color{gray},breaklines=true,numbers=left,]
auto lo
iface lo inet loopback

auto eth0
iface eth0 inet static
        address 10.1.4.2
        netmask 255.255.255.0
        network 10.1.4.0
        broadcast 10.1.4.255
        gateway 10.1.4.1
\end{lstlisting}

\begin{lstlisting}[frame=single,caption=/etc/resolv.conf,backgroundcolor=\color{gray},breaklines=true,numbers=left,]
nameserver 145.100.96.11
nameserver 145.100.96.22
\end{lstlisting}

\begin{lstlisting}[frame=single,caption=httperf and autobench,backgroundcolor=\color{gray},breaklines=true,numbers=left,]
# apt-get install build-essentials gawk httperf
# wget http://www.xenoclast.org/autobench/downloads/autobench-2.1.2.tar.gz
# tar zxvf autobench
# cd autobench-2.1.2
# make
# make install
# cd ..
# sed -i 's/echo set data style linespoints >> gnuplot.cmd/echo set style data linespoints >> gnuplot.cmd/' /usr/local/bin/bench2graph

# wget http://sourceforge.net/projects/gnuplot/files/gnuplot/4.6.1/gnuplot-4.6.1.tar.gz/download
# tar zxvf download
# cd gnuplot-4.6.1
# ./configure
# make
# make install
\end{lstlisting}
\end{document}

