\emph{Httperf} is a web server performance measurement tool developed by David Mosberger~\cite{mosberger1998httperf} at Hewlet-Packard Research Labs\footnote{\url{http://www.hpl.hp.com/research/linux/httperf/}}. It supports both HTTP/1.1 and SSL and aims to be robust enough to generate server overload.

\emph{Autobench}\footnote{\url{http://www.xenoclast.org/autobench/}} is a Perl script that wraps around \emph{Httperf}. It aims at automating the benchmarking process and offers extensive options, amongst which the ability to stepwise increase concurrent connections or the number of request. This results in graphs that visualize bottlenecks/tipping points.

In the course of this project \emph{Autobench} has been used to confirm and visualize the threshold of number of concurrent connections that naxsi can handle (as described in section~\ref{sec:Baseline performance measurement}). Details like the used parameters can be found in Appendix~\ref{sec:Measurement results}. To automate series of tests for script was written. It is explained separately in section {\color{red}{(PLEASE ADD REFERENCE)}}~\ref{sec:}.
