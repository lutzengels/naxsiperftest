\emph{Httperf} is a web server performance measurement tool developed by David Mosberger~\cite{mosberger1998httperf} at Hewlet-Packard Research Labs\footnote{\url{http://www.hpl.hp.com/research/linux/httperf/}}. It supports both HTTP/1.1 and SSL and aims to be robust enough to generate server overload. \emph{Autobench}\footnote{\url{http://www.xenoclast.org/autobench/}} is a Perl script that wraps around \emph{Httperf}. It aims at automating the benchmarking process and offers extensive options, amongst which the ability to stepwise increase concurrent connections or the number of requests per second. 

In the course of this project, \emph{Autobench} is used to confirm and visualize the threshold of the number of concurrent connections that Naxsi can handle (as described in section~\ref{sec:Baseline performance measurement}). The automation of incrementing the URL parameters is done by wrapping a script \footnote{\url{https://github.com/lutzengels/naxsiperftest/blob/master/tools/measure_naxsi_with_param_increments.pl}}  around Autobench.