\documentclass[Introduction]{subfiles}
\begin{document}
\section{Introduction}
\label{sec:Introduction}
More and more people will have access to the internet \footnote{\url{http://data.worldbank.org/indicator/IT.NET.USER.P2/countries?display=graph}} and they will have access to many web applications. Commercial companies have access to an ever growing market, but also criminals become more intelligent on how to reach end-users and abuse the web services users are accessing on a daily basis.
According to the web survey held by Netcraft in february 2013 \footnote{\url{http://news.netcraft.com/archives/category/web-server-survey/}, February 2013}, the majority of web servers is running Apache and it is followed by Microsoft. However, Nginx is gaining more popularity and it is not only used as a standard web server, but also as a reverse proxy for load balancing purposes. Naxsi (\emph{Nginx Anti Xss \& Sql Injection}) is developed as a response to common attacks that often occur on the internet. Naxsi is a firewall designed to work with Nginx. It works as a DROP-by-default firewall, and rules should be added to ACCEPT certain traffic. This concept is also known as whitelisting.
This paper focusses on the performance of an Naxsi-Nginx server when it is under heavy load and it has to deal with different attacks. 

\subsection{Research question}
\emph{How does a webserver perform with the use of Naxsi as the front-end firewall compared to a webserver without a Naxsi firewall?}

\subsubsection{Sub questions}

\end{document}
