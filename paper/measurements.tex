% autobench --single_host --host1 with_naxsi.test.nl --uri1 /index.php --low_rate 1 --high_rate 130  --rate_step 5 --const_test_time 30  --file results.csv

\documentclass[Measurements]{subfiles}
\begin{document}
\section{Measurements}
\label{sec:Measurements}
Basically, there are two main different scenarios for measuring the performance. The first one is where no application processing takes place, but the request with directly be answered with a HTTP 200 OK message. This creates the lowest overhead for Nginx and makes it possible to isolate the Naxsi performance overhead. The second one is a more real-life scenario, in which a Wordpress is setup on a back-end server. The front-end Nginx server acts as a reverse-proxy and only forwards the request to the appropriate server.

\subsection{Experiment 1: Nginx without Naxsi}

\subsection{Experiment 2: Nginx with Naxsi}

\subsection{Experiment 3: Nginx without Naxsi and Wordpress}

\subsection{Experiment 4: Nginx with Naxsi and Wordpress}
\end{document}

