\documentclass[Measurements]{subfiles}
\begin{document}
\section{Measurements}

\label{sec:Measurements}
Basically, there are two main different scenarios for measuring the performance. The first one is by measuring the performance of Nginx when Naxsi is disabled. The second one is by measure the performance when Naxsi is enabled.  In the first measurement scenario, the performance is measured as baseline. First a baseline measurement is performed with a Wordpress website on a back-end server. Second, the performance a baseline is measured where Nginx only returns a \verb+HTTP 200 OK+ response. This latter baseline measurement gives the lowest overhead and maximum performance of the Nginx webserver.

\subsection{Baseline performance measurement}
The baseline performance measurements are used as a reference for comparing later measurement results when Naxsi is enabled.

\subsubsection{Wordpress}
\label{sec:Baseline performance measurement}
In order to measure the performance of Naxsi, it is important to know the bottleneck of the back-end server doing all the processing at the application layer. Figure \ref{fig:Baseline measurement 1} shows that $\approx 21\%$ of the 10,000 requests produce a \verb+HTTP 5xx+ response when using 250 concurrent connections. Details can be found in appendix \ref{sec:baseline_measurement_1}.

\begin{table}[H]
\caption{Baseline measurement 1}
\begin{tabular}{|p{2cm}|p{1,5cm}|p{1,5cm}|p{1,5cm}|p{1,5cm}|p{1,5cm}|}
\hline
 & \textbf{1xx} & \textbf{2xx} & \textbf{3xx} & \textbf{4xx} & \textbf{5xx} \\ \hline
\textbf{Connections} & 0 & 0 & 8542 & 0 & 1458 \\ \hline
\end{tabular}
\label{fig:Baseline measurement 1}
\end{table}

By using a half-interval search method, the optimal number of concurrent connections, which gave the highest rate of 190 request per second. As can be seen in figure \ref{fig:Baseline measurement 2}, the total number of \verb+HTTP 3xx+ responses is 10,000. Details can be found in appendix \ref{sec:baseline_measurement_2}

\begin{table}[h]
\caption{Baseline measurement 2}
\begin{tabular}{|p{2cm}|p{1,5cm}|p{1,5cm}|p{1,5cm}|p{1,5cm}|p{1,5cm}|}
\hline
 & \textbf{1xx} & \textbf{2xx} & \textbf{3xx} & \textbf{4xx} & \textbf{5xx} \\ \hline
\textbf{Connections} & 0 & 0 & 10000 & 0 & 0 \\ \hline
\end{tabular}
\label{fig:Baseline measurement 2}
\end{table}

Based on the values derived from the measurement above, the performance is measured by stepping through the concurrent connections from 1 to 190. The response time is calculated by taking the average of each measurement of  each step. Each step has a measurement duration of 60 seconds. 

% autobench --single_host --host1 wp_without_naxsi.test.nl --uri1 /index.php --low_rate 1 --high_rate 230 --rate_step 1 --num_call 1 --const_test_time 60 --file results.csv
\begin{figure}[H]
\caption{Autobench baseline performance measurement}
\centering
\includegraphics[scale=0.55] {images/results/baseline_wp/output.png}
\label{fig:Baseline performance measurement}
\end{figure}

When staying under $\approx 190$ concurrent connections, the response time is \mbox{$\approx 30 ms$}. Figure \ref{fig:Baseline Nginx CPU usage} shows a steady increase of the number of jiffies when the numbers of concurrent connections increase. The number of jiffies only increase when the number of concurrent connections is still below 190. As soon as the concurrency raises above 190, the Nginx server starts returning \verb+HTTP 5xx+ error codes, because the back-end is not able to handle all requests. The memory foot print of Nginx is little, as can be seen in figure \ref{fig:Baseline Nginx memory usage}. The number of request have only a minimal impact on the memory usage. The number Bytes written on disk is also very minimal, as can be seen in figure \ref{fig:Baseline Nginx disk traffic}. Around 21.30 hours the disk IO drops slighty, which is the time the measurement ended. Figure \ref{fig:Baseline Nginx interface traffic} also shows that there is little network traffic. The number of received bits increments linearly as expected, because of incremental number of request received from server04.

\subsubsection{HTTP 200 OK}

Optimal number of concurrent connections $ \leftarrow MAX(AVG(r/s)) = 380$

\begin{figure}[H]
\caption{HTTP 200 OK baseline requests per seconds related to number of concurrent connections}
\centering
\includegraphics[scale=0.55] {images/results/baseline_200/output.png}
\label{fig:Baseline performance measurement}
\end{figure}

\subsection{Performance measurement with Naxsi}
Because earlier tests did not give conclusive results, the next step is to take the back-end application server out of the equation and focus completely on the Nginx and Naxsi server. Because the load on the Nginx server is still very low when the back-end server reaches its maximum performance, it is hard to measure the overhead Naxsi has on the server.

To test the performance of Naxsi, it is important to have a new baseline measurement. This time the client's request is not forwarded to the back-end server, but instead, Nginx returns a \verb+HTTP 200 OK+ response to the client. First, the Nginx server is testen without Naxsi enabled. Second, Naxsi is enabled the performance is measured again. Because a simple HTTP return code produces very little overhead for the server and the bandwith, it is not the response time that is compared, but the number of requests per second that the server is able to handle.

\subsubsection{Wordpress}

\paragraph{Allowed parameters}

\paragraph{Disallowed parameters}

% autobench --single_host --host1 wp_with_naxsi.test.nl --uri1 / --low_rate 1 --high_rate 230 --rate_step 1 --num_call 1 --const_test_time 60  --file results.csv
\begin{figure}[H]
\caption{Wordpress with Naxsi enabled}
\centering
\includegraphics[scale=0.55] {images/results/wp_with_naxsi_incremented_disallowed_parameters/output.png}
\label{fig:Baseline performance measurement}
\end{figure}

\subsubsection{HTTP 200 OK}

\paragraph{Allowed parameters}

\begin{figure}[H]
\caption{HTTP 200 OK baseline requests per seconds related to number of URL parameters}
\centering
\includegraphics[scale=0.55] {images/results/200_with_naxsi_incremented_allowed_parameters/output.png}
\label{fig:Baseline performance measurement}
\end{figure}

\paragraph{Disallowed parameters}

\begin{figure}[H]
\caption{HTTP 200 OK baseline requests per seconds related to number of URL parameters}
\centering
\includegraphics[scale=0.55] {images/results/200_with_naxsi_incremented_disallowed_parameters/output.png}
\label{fig:Baseline performance measurement}
\end{figure}


\end{document}

